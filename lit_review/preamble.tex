% basics
\usepackage[utf8]{inputenc}
\usepackage[T1]{fontenc}
\usepackage{sidenotes}
\usepackage{textcomp}
\usepackage{url}
\usepackage{hyperref}
\usepackage{xcolor}
\usepackage{titlesec}
\hypersetup{
  colorlinks,
  linkcolor={black},
  citecolor={black},
  urlcolor={blue!80!black}
}
\usepackage{amsmath, amsfonts, mathtools, amsthm, amssymb}

\newcommand\independent{\protect\mathpalette{\protect\independenT}{\perp}}
\def\independenT#1#2{\mathrel{\rlap{$#1#2$}\mkern2mu{#1#2}}}

% theorems
\usepackage{thmtools}
\usepackage[framemethod=TikZ]{mdframed}
\mdfsetup{skipabove=1em,skipbelow=0em, innertopmargin=5pt, innerbottommargin=6pt}

\theoremstyle{definition}

\declaretheorem[numberwithin=section, name=Definition]{definition}
\declaretheorem[numberwithin=section, name=Proposition]{prop}
\declaretheorem[numberwithin=section, name=Theorem]{theorem}
\declaretheorem[numberwithin=section, name=Lemma]{lemma}
\declaretheorem[numbered=no, name=Corollary]{corollary}
\declaretheorem[numbered=no, name=Identity]{identity}
\declaretheorem[numberwithin=section, name=Exercise]{exc}
\declaretheorem[numberwithin=section, name=Solution]{solution}
\declaretheorem[numbered=no, name=Remark]{remark}
\declaretheorem[numbered=no, name=Example]{example}


